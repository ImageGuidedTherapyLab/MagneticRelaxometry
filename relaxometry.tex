%
%  This simple example illustrates how documents can be
%  split into smaller segments, each segment processed
%  by latex2html separately.  This document can be
%  processed through latex and latex2html with the
%  corresponding makefile.
%

\documentclass{article}         % Must use LaTeX 2e
\usepackage[plainpages=false, colorlinks=true, citecolor=black, filecolor=black, linkcolor=black, urlcolor=black]{hyperref}		
%\usepackage{html}
\usepackage[left=.5in,right=.5in,top=.5in,bottom=.5in]{geometry}
\usepackage{makeidx,color,boxedminipage,verbatim}
\usepackage{wrapfig,graphicx,float}
\usepackage{amsmath,amsthm,amsfonts,amscd,amssymb} 
\usepackage{algorithm,algorithmic}

%%%%%%%%%%%%%%%%%%%%%%%%%%%%%%%%%%%%%%%%%%%%%%%%%%%%%%%%%%%%%%%%%%%%%%
%	Some math support.					     %
%%%%%%%%%%%%%%%%%%%%%%%%%%%%%%%%%%%%%%%%%%%%%%%%%%%%%%%%%%%%%%%%%%%%%%
%
%	Theorem environments (these need the amsthm package)
%
%% \theoremstyle{plain} %% This is the default

\newtheorem{thm}{Theorem}[section]
\newtheorem{cor}[thm]{Corollary}
\newtheorem{lem}[thm]{Lemma}
\newtheorem{prop}[thm]{Proposition}
\newtheorem{ax}{Axiom}

\theoremstyle{definition}
\newtheorem{defn}{Definition}[section]

\theoremstyle{remark}
\newtheorem{rem}{Remark}[section]
\newtheorem*{notation}{Notation}
\newtheorem*{exrcs}{Exercise}
\newtheorem*{exmple}{Example}

%\numberwithin{equation}{section}


%%%%%%%%%%%%%%%%%%%%%%%%%%%%%%%%%%%%%%%%%%%%%%%%%%%%%%%%%%%%%%%%%%%%%%
%	Macros.							     %
%%%%%%%%%%%%%%%%%%%%%%%%%%%%%%%%%%%%%%%%%%%%%%%%%%%%%%%%%%%%%%%%%%%%%%
%
%	Here some macros that are needed in this document:
\newcommand{\picdir}{./pdffig}
\newcommand{\rffrph}{\left(\delta \omega t + \varphi\right)}
\newcommand{\intphse}{\left( \int_0^t \Delta \omega(\tau) d\tau \right)}
\newcommand{\eqn}[1]{(\ref{#1})}
%
\title{Nanoparticle Magnetic Relaxometry for Cancer Detection}
\author{}
%%%%%%%%%%%%%%%%%%%%%%%%%%%%%%%%%%%%%%%%%%%%%%%%%%%%%%%%%%%%%%%%%%%%%%
\begin{document}                % The start of the document
\maketitle
\begin{abstract}                
Early detection is key to survival for cancer patients.  Screening
is the best way to catch cancer in its early stages, but for many anatomical
sites no screening technique is available.  Current screening techniques can
often yield false positives or inconclusive results, which lead to
unnecessary invasive procedures and treatment. Additionally, even the best
methods can only detect tumors as small as 2 mm in diameter, which is
already >10 million cells.  In an effort to provide clinicians with a
reliable non-invasive method of detecting cancer in its very early stages, a
system that uses the magnetic relaxometry properties of nanoparticles has
been developed to detect and count clusters of cancer cells as few as
20,000.  For most cancer patients, detection this early could mean the
difference between successful and unsuccessful treatment.  
\end{abstract}                

\paragraph{Overall Goals:}  
There are many questions to be answered regarding this new system.  Below
are project goals that are needed to translate this technology into the
clinical setting.
\begin{enumerate}
\item	Develop and calibrate a model for detecting the number and location
of nanoparticles in a sample by solving an inverse problem set with known
sources.
\item	Determine the optimal biokinetic model to describe the nanoparticle
interactions with different cell lines.
\item	Evaluate the feasibility and resolution limitations of imaging
disease with magnetic relaxometry.
\item	Differentiate between magnetic relaxometry signals from
nanoparticles bound to different targets.
\end{enumerate}

\paragraph{Action Items:}  
\begin{itemize}
\item Develop a mathematical model of the Relaxometry detection system from 
first principles \cite{Flynn2005}. The signal relaxation is assumed of the form:
\[
  F(t) = a_0 + a_1 \ln\left(1 + \frac{a_2}{t}\right) +a_3 \exp\left(-a_4 \; t\right)
\]
\item Acquire Phantom data in 1, 2, 3, 4, 5 point sources.  Build a data
dictionary of signals from combinatorics of sources and source locations. 
   \begin{itemize}
   \item Apply unsupervised machine learning (clustering algorithms) 
         to classify the signal from 1, 2, 3, 4, or 5 
         sources at different locations~\cite{Fraley2002,Sebastiani2003,Kapp2007,
         Shi2014,Criminisi2013,Murphy2012a,Bishop2006,Gelman2007}. 
         Ie how ill-posed is the problem ? which
         combinations of sources and source locations cluster together ? 
   \end{itemize}
\item Validate phantom measurements using deterministic quasi-newton curve fit of math 
      model~\cite{Fegan2010,Adolphi2010,Hathaway2011,Tessier2012,Adolphi2014,Schwindt2013,Shen2012,Hajdu2013,Paik2013,Amiri2011}. 
      Use initial guess as centroid of machine.
\item Compare deterministic inverse problem to advanced methods:
\begin{enumerate}
\item Global optimization using dictionary. 
   \begin{itemize}
   \item in-silico simulate 1, 2, 3, 4, 5 point sources. Build a dictionary of 
      signals from combinatorics of sources and source locations. 
   \item Code signal forward projection operator as a GPU kernel call from MATLAB
   \end{itemize}
\item Gaussian Process model~\cite{Rasmussen2006} with phantom data as training data
   \begin{itemize}
   \item Formulate Gaussian Process model within a physics based
         optimization framework~\cite{Constantinescu2013} , ie
         constrain the optimization by the physics model
   \item Solve using MCMC~\cite{Liu2002,Martin2012,kaipio2005statistical,Tan2012}
   \end{itemize}
\end{enumerate}
\item Evaluate within the context of a classical statistics framework, ie \cite{Rosner2010} Ch 11.
\end{itemize}

\section{Magnetostatics, Hard Ferrormagnets (M given and J=0)}
%%%%%%%%%%%%%%%%%%%%%%%%%%%%%%%%%%%%%%%%%%%%%%%%%%%%%%%%%%%%%%%%%%%%%%
%%%%%%%%%%%%%%%%%%%%%%%%%%%%%%%%%%%%%%%%%%%%%%%%%%%%%%%%%%%%%%%%%%%%%%
Following Jackson~\cite{jackson1999classical}, `hard' ferromagnets,
having a magnetization that is essentially independent of the applied
fields may be treated as if they have a fixed magnetization $\vec{M}(x)$.
Guass law for magnetism becomes
\[
  \nabla \cdot \vec{B} = 
  \mu_0 \nabla \cdot (  \vec{H} + \vec{M}) 
\]
Given the magnetic scalar potential 
\[
   \nabla \Phi_M \equiv \vec{H}
\]
and the effective magnetic-charge density
\[
   \rho_M \equiv - \nabla \cdot  \vec{M}
\]
Gauss law reduces to a poisson equation
\[
   \nabla^2 \Phi_M = \rho_M 
\]
The Greens function solution to this equation is
\[
   \Phi_M = \int G(x,y) \rho_M(y) dy 
   \qquad
   G(x,y) = \frac{1}{4 \; \pi \; r}
\]
Under the assumption that the 
effective magnetic-charge density
is of the form of the divergence of a point source magnetic dipole,
$\vec{\mu}\in \mathbb{R}^3$
\[
   \rho_M  (x) = \nabla_x \cdot \vec{\mu} \delta(x - x_0)
\]
The solution must be interpreted in the sense of distributions
and reduces to the classical dipole equations
\[
\begin{split}
   \Phi_M & = \int G(x,y) ( \nabla_y \cdot \vec{\mu} \delta(y - x_0) ) dy 
    \\
          & = \int
               \left(
                \frac{\partial}{\partial y_1} G(x,y) \mu_1 \delta(y - x_0) 
                    +
                \frac{\partial}{\partial y_2} G(x,y) \mu_2 \delta(y - x_0) 
                    +
                \frac{\partial}{\partial y_3} G(x,y) \mu_3 \delta(y - x_0) 
               \right)
               dy 
    \\
          &  = \nabla_x G(x,x_0)  \cdot \vec{\mu} 
    \\
          & = \frac{\vec{r} \cdot \vec{\mu} }{ 4 \; \pi \; r^3}  
\end{split}
\]
\[
  \vec{B} = \mu_0 \nabla \Phi_M  =  \frac{\mu_0 }{ 4\pi} 
      \left( 
   \frac{ 3 \vec{r} (\vec{r} \cdot \vec{\mu})}{r^5}  - \frac{\vec{\mu}}{r^3}
     \right)
\]

%%%%%%%%%%%%%%%%%%%%%%%%%%%%%%%%%%%%%%%%%%%%%%%%
%%%%%%%%%%%%%%%%%%%%%%%%%%%%%%%%%%%%%%%%%%%%%%%%
%%%%%%%%%%%%%%%%%%%%%%%%%%%%%%%%%%%%%%%%%%%%%%%%
\nocite{*}
\bibliographystyle{plainnat}  % Here the bibliography
\bibliography{relaxometry}

\end{document}
