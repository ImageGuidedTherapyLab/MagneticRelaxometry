\documentclass[a4paper]{article}

% ---------- packages
\usepackage[english]{babel}
\usepackage[utf8x]{inputenc}
\usepackage{amsmath}
\usepackage[left=.8in,right=.8in,top=.8in,bottom=.8in]{geometry}
\usepackage{color,graphicx}
\usepackage{float}
\usepackage{amssymb}
\usepackage{algorithm,algorithmic}

\newenvironment{definition}[1][Definition]{{\scriptsize\marginpar{#1}}\begin{trivlist}
\item[\hskip \labelsep {\bfseries (Definition) #1}]}{\end{trivlist}}
% ---------- to add line break in table
\newcommand{\specialcell}[2][c]{%
  \begin{tabular}[#1]{@{}c@{}}#2\end{tabular}}
\newcommand{\picdir}{./pdffig}

% ---------- Title
\title{Linearization of the Biot Savart Law}
\author{Sara Loupot\\
        David Fuentes}
\date{May 23, 2015}


\begin{document}

\maketitle
\section{Introduction}
The Biot Savart Law describes the magnetic field as a function of distance from a magnetic dipole source.    This relationship is used in magnetic relaxometry to relate the signal received by the SQUID pickup coils to the location of the bound nanoparticles.  In order to create an image, the magnetic inverse problem must be solved.  This problem is ill-posed: there are more unknown parameters than known.  To employ state of the art methods to attempt to solve this problem, the non-linear Biot Savart relationship must be made into a linear form, Ax=b.  Under some assumptions, this can be done.

\section{The Biot Savart Law}
The Biot Savart Law states

\begin{equation}
\centering
  \vec{B}(\vec{r},\vec{m}) = \frac{\mu_{0}}{4\pi}
    \left[
      \frac{3(\vec{m}*\vec{r})\vec{r}}{r^{5}}-\frac{\vec{m}}{r^{3}}
    \right]
  \label{eq:BiotSavart}
\end{equation}
Where $\vec{B}$ is the magnetic field at a point $\vec{r}$ from a dipole
$\vec{m}$.  Immediately after the field is turned off, the moment vector can
be approximated as only having a non-zero longitudinal component, $\vec{m}
\approx m\hat{z}$, which simplifies the dot product:
\[\vec{m}*\vec{r}= m_{i}r_{i} = m r_{z}\]
and the second term:
\[\frac{\vec{m}}{r^{3}} = \frac{m }{r^{3}}\hat{z}\]
The SQUIDs only detect $\vec{B}$ parallel to their axis.  If we assume they are arranged parallel to the $z$ axis then

\begin{equation}
\centering
\vec{B}(\vec{r},\vec{m}) = (\vec{B}(\vec{r}))_{z} = \frac{\mu_{0}}{4\pi}
    \left[\frac{3m r_{z}}{r^{5}}r_{z} - \frac{m }{r^{3}}
    \right]\hat{z}
\end{equation}
which allows us to factor out $m $:
\begin{equation}
\vec{B}(\vec{r},\vec{m})=\frac{\mu_{0}}{4\pi}m [\frac{r_{z}^{2}}{r^{5}}-\frac{1}{r^{3}}]\hat{z}
\end{equation}
Now, $B$ is separable into  independent functions that depend on the moment
and the distance.
\begin{equation}
\vec{B} = f(m )g(\vec{r}) 
\end{equation}
where 
\begin{equation}
f(m ) = \frac{\mu_{0}}{4\pi}m 
\end{equation}
and 
\begin{equation} \label{eq:g}
g(\vec{r})=\frac{r_{z}^{2}}{r^{5}}-\frac{1}{r^{3}}
\end{equation}

\subsection{Discretization}

We will assume a Galerkin discretization of our solution field, $m_z(r)$.
\[
  m(r) = \sum_j m_j \phi_j  (r)
   \qquad 
  \phi_j = 
   \left\{ \begin{split}
     1, \quad r \in    \Omega_j \\
     0, \quad r \notin \Omega_j
   \end{split}
   \right.
\]
Here the imaging domain, $\Omega$, is decomposed into disjoint pixels,
$\Omega_j$
\[
   \Omega = \cup_j \Omega_j
  \qquad
   \Omega_i  \cap \Omega_j = \emptyset
\]


Finally, the field detected by sensor $i$ is a linear combination of the
field produced by each source $m_j$ 
located at $\vec{r}_{ij}$ from
the sensor. 
 \[
\vec{B}_i(m)=\frac{\mu_{0}}{4\pi}
  \left(\sum_j m_j \phi_j  (\vec{r}_{ij})\right)
  \underbrace{
   \left[
       \frac{r_{z_{ij}}^{2}}{r^{5}_{ij}}-\frac{1}{r^{3}_{ij}}
   \right]
   }_{g(\vec{r}_{ij})}
 \hat{z}
\]
We then assume a discretized field of view with $j$ pixels, each with their own moment $m_{j}$ a distance $\vec{r}_{ij}$ from sensor $i$.  To construct the linear problem, define a $n\times1$ vector $\vec{B}$ of the field detected by sensor $i$ where $n$ is the number of sensors, a $m\times1$ vector $\vec{x}$ of the source from pixel $j$ where $m$ is the number of pixels, and a $m\times n$ matrix $A$ of $g(\vec{r}_{ij})$ where $\vec{r}_{ij}$ is the vector from pixel $j$ to sensor $i$, as defined in equation \ref{eq:g}.

\begin{equation}
\begin{bmatrix}
B_{1}\\  
\vdots\\ 
B_{i}\\
\vdots\\
B_{n}
\end{bmatrix}
=
\begin{bmatrix} \label{eq:fullMatrix}
g(\vec{r}_{1,1}) & g(\vec{r}_{1,2}) & \dots  & g(\vec{r}_{1,m}) \\
g(\vec{r}_{2,1}) & g(\vec{r}_{2,2}) & \dots & g(\vec{r}_{2,m})\\
\vdots & \vdots & \ddots & \vdots\\
g(\vec{r}_{n,1}) & g(\vec{r}_{n,2}) & \hdots & g(\vec{r}_{n,m})\\
\end{bmatrix} 
\begin{bmatrix}
m_{1} \\ \vdots \\ m_{j} \\ \vdots \\ m_{m}
\end{bmatrix}
\qquad
\Leftrightarrow
b = A \; x
\end{equation}
Since the location of the detectors with respect to the field of view grid points is defined, $\vec{x}$ is the only unknown.  

Columns of the matrix, $A$, may be interpreted as 'distance' vectors  from
magnetic source $j$ to each sensor.
\[
A = 
\begin{bmatrix}
 \vec{g}_1
 &
 \vec{g}_2
 &
 ...
 &
 \vec{g}_m
\end{bmatrix}
\qquad
\vec{g}_j
 = 
\begin{bmatrix} 
g(\vec{r}_{1,j}) & \\
g(\vec{r}_{2,j}) & \\
\vdots           & \\
g(\vec{r}_{n,j}) & \\
\end{bmatrix} 
\]


\subsection{Bias Correction }
The bias correction of \cite{Gorodnitsky} is of the form
 \[
\vec{B}_i(m)=\frac{\mu_{0}}{4\pi}
  \left(\sum_j  \frac{m_j}{\sqrt{\sum_{k=1}^n  g^2 (\vec{r}_{kj})}}  \phi_j(\vec{r}_{ij})\right)
  \underbrace{
   \left[
       \frac{r_{z_{ij}}^{2}}{r^{5}_{ij}}-\frac{1}{r^{3}_{ij}}
   \right]
   }_{g(\vec{r}_{ij})}
 \hat{z}
\]
The artificial term, $\sqrt{\sum_{k=1}^n  g^2 (\vec{r}_{kj})}$,
represents the average distance of magnetic source $j$ to the sensors and allows numerical techniques more preferentially weight terms further from the sensor.
\textit{\color{red} @saraloupot
Is the physics is preserved ? NEED DIMENSIONAL ANALYSIS.
units of $m$ must be changed to match the b-field. } 
\begin{equation}
\vec{B}_i \left[\frac{N}{Am}\right] = \frac{\mu_{0}}{4\pi} \left[\frac{N}{A^{2}}\right] \left(\sum_j  \frac{m_{j} [Am^2]}{\sqrt{\sum_{k=1}^n  g^2 (\vec{r}_{kj})}}  \phi_j(\vec{r}_{ij})\right)g(\vec{r}_{ij})\left[\frac{1}{m^3}\right]
\end{equation}
\begin{equation}
\left[\frac{N}{Am}\right] = \left[\frac{N}{A^2}\right]\left[A m^2\right]\left[\frac{1}{m^3}\right]
\end{equation}
\[
  r \quad \uparrow 
\qquad \Rightarrow \qquad
  g(r) \quad \downarrow 
\qquad \Rightarrow \qquad
  \frac{1}{g(r)} \quad \uparrow 
\]
The augmented equations is now of the form
\[
b = \hat{A} \; \hat{x}
\]
where columns of the augmented system matrix are unit normalized
\[
\hat{A} = 
\begin{bmatrix}
 \frac{\vec{g}_1}{\|\vec{g}_1\|}
 &
 \frac{\vec{g}_2}{\|\vec{g}_2\|}
 &
 ...
 &
 \frac{\vec{g}_m}{\|\vec{g}_m\|}
\end{bmatrix}
\qquad
\vec{g}_j
 = 
\begin{bmatrix} 
g(\vec{r}_{1,j}) & \\
g(\vec{r}_{2,j}) & \\
\vdots           & \\
g(\vec{r}_{n,j}) & \\
\end{bmatrix} 
\]


%%%%%%%%%%%%%%%%%%%%%%%%%%%%%%%%
\section{Linear Inverse solve}
%%%%%%%%%%%%%%%%%%%%%%%%%%%%%%%%
It would seem that the solution is trivial, with a simple linear inverse solve $x=A^{-1}b$.  However, since $n<<m$ the problem is still ill-posed.  There are an infinite number of exact solutions.  A simple inverse solution results in a solution of a source under each detector proportional to the strength of the signal received from that detector.  Clearly this is a viable solution, but it is unlikely to be the true solution.  

%%%%%%%%%%%%%%%%%%%%%%%%%%%%%%%%
\section{$l_1$ norm minimization}
%%%%%%%%%%%%%%%%%%%%%%%%%%%%%%%%
In the MRX application, we can assume that the true solution will be sparse:
only a few voxels will have a non-zero contribution to the net moment.  To
increase the chances of converging to the true solution, we can attempt to
find the minimum number of sources that still solve Equation
\ref{eq:fullMatrix} exactly.  The $l_p$ norm can be used to define sparsity with parameter p.
 \begin{equation}
 l_{p}(\vec{x})=\left[\sum_{i}x_{i}^{p}\right]^{1/p} 
 \end{equation}
When p=2, the $l_{p}$ norm is the magnitude of the vector $\vec{x}$.  When p=1, $l_{p}$ is the sum of the components of $\vec{x}$.  When p=0, $l_{p}$ is the number of non-zero entities of $\vec{x}$.  Ideally, to find the sparsest solution that satisfies the equality constraint, it would be best to minimize the $l_{p}$ norm.  Since the $l_0$ norm is not convex, it is difficult to compute explicitly. Instead, we must maximize sparsity by minimizing the L1 norm of $x$ while applying the condition that $Ax = b$.
\begin{equation}
\centering
  \begin{array}{ll@{}r@{}r@{}l}
    \text{min} & \sum\limits_{i} |x_i| \\[\jot]
    \text{s.t.}& Ax = b \\
    & \multicolumn{4}{l}{x \geq 0}
  \end{array}
\end{equation}
We will solve this as a sequence of unconstrained Quadratic
Penalty problems with a non-smooth objective function
\[
\min_x \|\Phi x\|_1 + \frac{1}{2 \mu}\|Ax-b\|_2^2
\]
For large $\mu$, the penalty function does not accurately enforce the
constraint. As we have seen, typical approaches increasingly impose the constraint
by letting $\mu \rightarrow 0$. 
An alternative is to apply the Augmented Lagrangian approach or the 
so-called Bregman iteration~\cite{goldstein2009split}
where in the penalty term constants is fixed and we more accurately solve
the constraint at each iteration. These approach introduce an
auxillary variable $w \in \mathbb{R}^n$ and enforce the transformation
to this variable.
\[
  \min_x \frac{1}{2} \|Ax -b\|_2^2 + \tau \|w\|_1
\qquad
\text{ such that}
\quad
\Phi x = w
\]
This may be interpreted in our previous notation as
\[
f(\hat{x}) = f(x,w) = \frac{1}{2} \|Ax -b\|_2^2 + \tau \|w\|_1
\qquad
\begin{bmatrix}
c_1 \\
c_2 \\
 .  \\
 .  \\
\end{bmatrix}
=
\Phi x - w
\]
and the Augmented Lagrangian may be written as
\[
\mathcal{L}_a( \underbrace{x,w}_{\hat{x}},\lambda;\mu)
=
\underbrace{
  \frac{1}{2} \|Ax -b\|_2^2 
+ \tau \|w\|_1
}_{f(\hat{x})}
-
\underbrace{
 ( \lambda ,\Phi x - w)
}_{ \sum_i \lambda_i c_i(\hat{x})}
+ 
\underbrace{
\frac{1}{2 \mu} \|(\Phi x - w)\|_2^2
}_{ \frac{1}{2 \mu} \sum_i c_i^2(\hat{x})}
\]
within the scope of Algorithm~\ref{AugmentedLagrangianApproach},
given $\lambda^k$ and $\mu^k$, the subproblem is still
formidable and contains L1 and L2 terms
\[
\min_{x,w}
         \mathcal{L}_a(\cdot, \lambda^k; \mu^k)
=
\min_{x,w}
  \frac{1}{2} \|Ax -b\|_2^2 
+ \tau \|w\|_1
- ( \lambda^k ,\Phi x - w)
+ \frac{1}{2 \mu }\|(\Phi x - w)\|_2^2
\]
The Lagrangian may be written in an equivalent form, 
using the linearity of the inner product
\[
\begin{split}
\frac{1}{2c} (a-cb , a-cb) 
& = 
\frac{1}{2c} \left[
             (a-cb , a)   
                   -
             (a-cb , cb)   
             \right]
= 
\frac{1}{2c} \|a\|^2
-
\frac{1}{2c} c(b,a)
-
\frac{1}{2c} c(a,b)
+
\frac{c^2}{2c} \|b\|^2
=
\frac{1}{2c} \|a\|^2
-
(a,b)
+
\frac{c}{2} \|b\|^2
\\
& \Rightarrow
\qquad
\frac{1}{2c} \|a-cb\|^2
-
\frac{c}{2} \|b\|^2
=
\frac{1}{2c} \|a\|^2
-
(a,b)
\end{split}
\]
and minimizing with respect to $(x,w)$,
$ \frac{\mu}{2  }\| \lambda^k \|_2^2$  may be considered a constant
\[
\min_{x,w}
         \mathcal{L}_a(\cdot, \lambda^k; \mu^k)
=
\min_{x,w}
  \frac{1}{2} \|Ax -b\|_2^2 
+ \tau \|w\|_1
+ \frac{1}{2 \mu }\|\Phi x   - w - \mu \lambda^k \|_2^2
- \frac{\mu}{2  }\| \lambda^k \|_2^2
\quad
\Leftrightarrow
\quad
\min_{x,w}
  \frac{1}{2} \|Ax -b\|_2^2 
+ \tau \|w\|_1
+ \frac{1}{2 \mu }\|\Phi x   - w - \mu \lambda^k \|_2^2
\]
The approach becomes tractable if we break the subproblem
up into an L1 subproblem and L2 subproblem using
an alternating direction or coordinate descent techinque.
%\[
%\begin{split}
%  w^k, \lambda^k \text{ fixed }
% \qquad
%&\min_x 
%  \frac{1}{2} \|Ax -b\|_2^2 
%- (\lambda^k ,\Phi x - w^k)
%+ \frac{1}{2 \mu }\|\Phi x - w^k\|_2^2
%\\
%  x^k, \lambda^k \text{ fixed }
%\qquad
%& \min_w 
%  \tau \|w\|_1
%- (\lambda^k ,\Phi x^k - w)
%+ \frac{1}{2 \mu }\|\Phi x^k - w\|_2^2
%\end{split}
%\]
\[
\begin{split}
\underbrace{\text{ L2 subproblem }}_{w^k , \lambda^k \text{ fixed}}
\qquad
&\min_x 
  \frac{1}{2} \|Ax -b\|_2^2 
+ \frac{1}{2 \mu }\|\Phi x   - w^k - \mu \lambda^k \|_2^2
\\
\underbrace{\text{ L1 subproblem }}_{x^k , \lambda^k \text{ fixed}}
\qquad
& \min_w 
  \tau \|w\|_1
+ \frac{1}{2 \mu }\|\Phi x^k - w   - \mu \lambda^k \|_2^2
\end{split}
\]
A summary of the Augmented Lagrangian Algorithm in this context is presented in 
Algorithm \ref{AugmentedLagrangianApproachCompressedSensing}.
Notice that we have converted the constrained problem
into two unconstrained problems. 

\begin{algorithm}                      % enter the algorithm environment
\caption{Augmented Lagrangian Approach For CS}% give the algorithm a caption
\label{AugmentedLagrangianApproachCompressedSensing}               % and a label for \ref{} commands later in the document
\begin{algorithmic}                    % enter the algorithmic environment
        \STATE
\[
  \min_{x,w} \frac{1}{2} \|H(x)\|_2^2 + \tau \| w\|_1
\qquad
\text{ such that}
\quad
\Phi x = w
\qquad
H(x) = Ax-b
\]
    \REQUIRE $\mu > 0$, tolerance $\tau > 0$, and initial guess $w^0=0$, and $\lambda^0=0$
    \WHILE{not converged, ie $\|\Phi x^k - w^k \| > \epsilon$ and $\|H(x^k)\|_2^2 > \epsilon$}
        \STATE Solve $L_2$ subproblem. 
             \[
                x^{k+1} = \min_x \frac{1}{2} \|H(x)\|_2^2  + \frac{1}{2\mu} \|\Phi x -w^k - \mu \lambda^k \|_2^2
             \]
         The derivative of this objective function for the subproblem $f_\text{L2sub}$ is given by
             \[
             \nabla f_\text{L2sub}
             =
             A^\top 
             \left(
             A x - b
             \right)
             + 
             \frac{1}{\mu} \Phi^\top 
             \left(
             \Phi x - w^k - \mu \lambda^k
             \right)
             \]
        \STATE Solve $L_1$ subproblem. 
             \[ \begin{split}
	        w^{k+1} = \min_w \tau \|w\|_1 & + \frac{1}{2\mu} \|\Phi x^{k+1} - w - \mu \lambda^k \|_2^2
                 \\
                 w_i & = \text{soft}_{\tau \mu} \left( (\Phi x^k)_i- \mu\lambda_i \right)
                 \\
                     &  \text{soft}_a(b) \equiv \frac{b}{|b|} \max (0,|b|-a)
             \end{split}\]
        \STATE Update Lagrange Multiplier.
             \[
                \lambda^{k+1} = \lambda^k -  \frac{1}{\mu} (\Phi x^{k+1} -w^{k+1})
             \]
    \ENDWHILE
\end{algorithmic}
\end{algorithm}

\paragraph{Derivatives}
The L2 subproblem may be solved
with a linesearch or trust region approach using finite difference or analytical derivatives.
For the general case
\[
\min_{\vec{x} \in \mathbb{R}^n}  f(\vec{x})
=
\min_{\vec{x} \in \mathbb{R}^n}  \frac{1}{2} \vec{h}^\top(\vec{x}) \vec{h}(\vec{x})
\qquad
\vec{h}(\vec{x}) = 
\begin{bmatrix}
A \vec{x} - \vec{b}
\\
\frac{1}{\sqrt{\mu}}
\left(
\Phi \vec{x} - \vec{w} - \mu \vec{\lambda} 
\right)
\end{bmatrix}
\]
The gradient of the objective function $f(\vec{x})$ is given 
as the matrix vector product of the Jacobian transpose times the
residual.
\[
\left(
\nabla f
\right)_i
=
\frac{\partial}{\partial x_i} 
\left( 
\frac{1}{2} 
\sum_l^m h_l h_l
\right) 
 = 
\left( 
\sum_l^m h_l 
\frac{\partial h_l}{\partial x_i} 
\right) 
\qquad
\nabla f = J^\top \vec{h}
\qquad
J_{ij}    
=
\frac{\partial h_i}{\partial x_j} 
\qquad
J
=
\begin{bmatrix}
A
\\
\frac{1}{\sqrt{\mu}}
\Phi
\end{bmatrix}
\]
Applying this formula to each term in our L2 subproblem yeilds
\[
\nabla
\left(
  \frac{1}{2} \|H(x)\|_2^2 
+ \frac{1}{2 \mu }\|\Phi x   - w^k - \mu \lambda^k \|_2^2
\right)
=
\begin{bmatrix}
A
\\
\frac{1}{\sqrt{\mu}}
\Phi
\end{bmatrix}^\top
\begin{bmatrix}
A \vec{x} - \vec{b}
\\
\frac{1}{\sqrt{\mu}}
\left(
\Phi \vec{x} - \vec{w} - \mu \vec{\lambda} 
\right)
\end{bmatrix}
=
A^\top 
\left(
A x - b
\right)
+ 
\frac{1}{\mu} \Phi^\top 
\left(
\Phi x - w^k - \mu \lambda^k
\right)
\]
\paragraph{Soft Thresholding Operator}
The L1 subproblem is (perhaps surprisingly) now given
by the soft thresholding operator
\[
  w_i = \text{soft}_{\tau \mu} \left( (\Phi x^k)_i- \mu\lambda_i \right)
\qquad
  \text{soft}_a(b) \equiv \frac{b}{|b|} \max (0,|b|-a)
\]
To see this, we need to generalize our definition
of the derivative for $\|\cdot\|_1$. Consider the subdifferentiable 
of the L1 problem
\[
0 
\quad \in \quad
  \tau \frac{w_i}{|w|_i}
- \frac{1}{\mu} 
\left(
(\Phi x^k ))_i
- w_i
- \mu \lambda^k_i
\right)
\]

\begin{figure}[h]
\centering
\begin{tabular}{cc}
\scalebox{0.31}{\includegraphics*{\picdir/SubdifferentialDef}}
&
\scalebox{0.41}{\includegraphics*{\picdir/SubdifferentialAbsValue}}
\\
(a) & (b) \\
\end{tabular}
\caption{Subdifferential.
(a) If $f$ is  differentiable, then the gradient is the  subgradient.
However, the subgradient may exist when the gradient does not exist.
In fact there may be several subgradients at this point.
(b) Absolute value. Consider $f(z) = |z|$. 
For $x < 0$ the subgradient is unique:
$\partial f(x) = \{-1\}$. 
Similarly, for $x > 0$ we have $\partial f(x) = \{1\}$. 
At $x = 0$ the subdifferential
is defined by the inequality
 $|z| > g z $ for all $z$, which is satisfied if and only if $ g \in
[-1, 1]$.  Therefore we have $\partial f(0) = [-1, 1]$.
}\label{SubdifferentialExample}
\end{figure}
\begin{definition}[Subdifferential]
We say a vector $g \in \mathbb{R}^n$ is a \textbf{subgradient}
of $f:\mathbb{R}^n \rightarrow \mathbb{R}$ at $x \in \mathbb{R}^n$ if
\[
   f(z) \geq f(x)  + g^\top (z-x)  \qquad \forall z
\]
The \textbf{set} of subgradients of $f$ at the point $x$  is called the
\textbf{subdifferential} at $x$ and is denoted $\partial f(x)$.
\[
  \partial f(x)  
\equiv
\left\{ g: 
   f(z) \geq f(x)  + g^\top (z-x)  \qquad \forall z
\right\}
\]
\end{definition}

To illustrate the solution to this equation
consider
\[
 0 \in a \text{ sign}(x) + x -b
\quad \Leftrightarrow \quad
 b \in a \text{ sign}(x) + x 
\qquad
 \text{ sign}(x) 
\equiv
\left\{
\begin{split}
-1       & \quad x<0
\\
(-1,1)   & \quad x=0
\\
1        & \quad x>0
\end{split}
\right.
\]

\begin{figure}[h]
\centering
\scalebox{0.41}{\includegraphics*{\picdir/l1subproblem}}
\caption{Solution to $L_1$ subproblem. (Provided by W. Stefan)}
\label{SolutionL1SubProblem}
\end{figure}

As seen in Figure~\ref{SolutionL1SubProblem},
when does $ b = x + a \; \text{sign}(x) $ ? 
\begin{itemize}
\item[Case I]
  \[
  b > a 
  \quad \Rightarrow \quad
  b = x + a
  \quad \Rightarrow \quad
  x = b -a
  \]
\item[Case II]
  \[
  |b| < a 
  \quad \Rightarrow \quad
   x = 0
  \]
\item[Case III]
  \[
  b < - a 
  \quad \Rightarrow \quad
  b = x - a
  \quad \Rightarrow \quad
  x = b + a
  \]
\end{itemize}

These three cases can be convienently expressed by the so-called
soft threshold operator. One may directly verify that:
\[
x = \text{soft}_a(b) = \frac{b}{|b|} \max (0,|b|-a) 
  = 
  \left\{
   \begin{split}
      b+a  & \quad  b     < -a \\
       0   & \quad |b| \leq -a \\
      b-a  & \quad  b     >  a \\
   \end{split}
  \right.
\]
The solution to the L1 problem is given by a change of
variables.

\section{Future Work}

From here we plan to investigate applying the FOCUSS algorithm to MRX image formation.  This will require improvements on the current assumptions, including a more accurate representation of the sensor geometry.  Different initial conditions will need to be investigated including using theoretical values and phantom data.  The regularization parameter will need to be optimized. The suitability of other algorithms such as MUSIC and BRAINSTORM will also be considered.


\begin{thebibliography}{2}
\bibitem{Gorodnitsky} Gorodnitsky, et. al. Neuromagnetic source imaging with FOCUSS: a recursive weighted minimum norm algorithm, Electroencepholography and clinical Neurophysiology 95 (1995) 231-251.

\bibitem[Goldstein and Osher, 2009]{goldstein2009split}
Goldstein, T. and Osher, S. (2009).
\newblock The split bregman method for l1 regularized problems.
\newblock {\em SIAM Journal on Imaging Sciences}, 2(2):323--343.

\end{thebibliography}



\end{document}
